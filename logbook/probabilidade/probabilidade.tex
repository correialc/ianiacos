\chapter{Probabilidade}

Probabilidade é a teoria matemática utilizada para se estudar a incerteza oriunda de fenômenos de caráter aleatório. Fornece a base para os métodos que utilizamos quando fazemos generalizações a partir de dados observados.

A \textbf{Probabilidade} em si é uma médida numérica (variando de 0 a 1) da possibilidade de um evento ocorrer. Ela estuda situações onde os resultados são variáveis:
\begin{itemize}
	\item Os resultados possíveis são conhecidos;
	\item Não se pode saber \textit{a priori} qual deles ocorrerá.
\end{itemize}

\textbf{Experimentos Aleatórios} são aqueles cujos resultados podem não ser os mesmos, ainda que sejam repetidos sob condições idênticas:
\begin{itemize}
	\item O resultado de um experimento aleatório específico não é previamente conhecido;
	\item Todos os possíveis resultados podem ser descritos.
\end{itemize}

\textbf{Espaço Amostral} ( \(\Omega\) ) é o conjunto de todos os possíveis resultados de um experimento aleatório:
\begin{itemize}
	\item \(\Omega_1 = \{ 1, 2, 3, 4, 5, 6 \} \) - Faces de um dado
	\item \(\Omega_2 = \{ t \in R \mid t \geq 0 \} \) - Vida útil de uma lâmpada
\end{itemize}

\textbf{Evento} é um subconjunto do espaço amostral:
\begin{itemize}
	\item \(A_1 = \{ 2, 4, 6 \} \) - Obter uma face par do dado
	\item \(B_2 = \{ t \geq 10000 \} \) - A lâmpada durar pelo menos 10000 minutos
\end{itemize}

\section{Axiomas de Probabilidade}

Dado um espaço amostral, \(\Omega\), suponha que estamos estudando um evento A. A
probabilidade do evento A ocorrer é denotada por P(A). A função P(A) só será uma
probabilidade se ela satisfaz três condições básicas:
\begin{itemize}
	\item \( 0 \leq P(A) \leq 1 \)
	\item \( P(\Omega) = 1 \)
	\item \( P(A_1 \cup A_2 \cup A_3...) = P(A_1)+P(A_2)+P(A_3)+...\), se os eventos \(A_1, A_2, A_3, ...\) forem disjuntos (mutuamente exclusivos).
\end{itemize}

\section{Atribuição de Probabilidade aos Elementos de \( \Omega \)}

Visão Clássica:

\[ P(A) = \frac{casos favoráveis}{total de casos} \]

Visão Frequentista:

\[ P(A) = \frac{ocorrências de A}{repetições do experimento} \]

Interseção de Eventos: 

\[ A \cap B \]

Eventos Disjuntos: 

\[ A \cap B = \varnothing \]

União de 2 Eventos:

\[ P(A \cup B) = P(A) + P(B) - P(A \cap B) \]  

Eventos Complementares:

\[ A \cup B = \Omega \] 
\[ A \cap B = \varnothing \]
\[ \bar{A} = B \]
\[ \bar{B} = A \]
\[ P(A) = 1 - P(B) \]
\[ P(\bar{A}) = P(B) = 1 - P(A) \]

\subsection{Probabilidade Condicional}

Probabilidade de um evento A ocorrer dado que um evento B já ocorreu:

\[ P(A \mid B) = \frac{P(A \cap B)}{P(B)} \]

A partir da qual é possível deduzir a \textbf{regra do produto condicional}:

\[ P(A \cap B) = P(A \mid B).P(B) \]

Dois eventos podem ser considerados independentes quando \( P(A \mid B) = P(A) \) e \( P(B \mid A) = P(B) \). É possível deduzir também que, para A e B serem independentes, \( P(A \cap B) = P(A).P(B) \). 

\section{Variáveis Aleatórias}

O resultado de um experimento probabilístico pode ser frequentemente uma contagem ou uma medida. Quando isso ocorre, o resultado é chamado de variável aleatória. A variável aleatória \textbf{x} representa um valor numérico associado a cada um dos resultados de um experimento probabilístico.

\section{Modelos Probabilísticos para Variáveis Aleatórias Discretas}

Uma variável aleatória \textbf{X} é discreta se assume valores (x) que podem ser contados, ou seja, se houver um número finito ou contável de resultados possíveis que possam ser enumerados.

\subsection{Distribuição Binomial}

Condições:
\begin{itemize}
	\item n tentativas independentes
	\item X pode assumir os valores 0, 1, 2,..., n
	\item Probabilidade de sucesso em n tentivas = p
	\item Probabilidade de fracasso em n tentativas = q
	\item p + q = 1
\end{itemize}

Nestas condições, o comportamento de X pode ser descrito pela \textbf{Distribuição Binomial} com parâmetro n. 

\begin{itemize}
	\item Notação: \(X \sim B(n; p) \)
	\item Média de X: \( E(x) = n.p \)
	\item Variância de X: \( \sigma^2 = n.p.q \), onde q = 1 -p
\end{itemize}

Função de probabilidade de \(X \sim B(n; p) \):

\[\beq{ P(X = x) = \binom{n}{x} p^x (1 - p)^{n-x} }\]  

Onde \(\beq{ \binom{n}{x} }\) representa o coeficiente binomial calculado por \(\beq{ \binom{n}{x} = \frac{n!}{x!(n-x)!} }\) 

\subsection{Distribuição Poisson}

Condições:
\begin{itemize}
	\item a variável aleatória (X) consiste na contagem de eventos discretos que ocorrem em um meio contínuo (tempo, volume, área, etc.);
	\item \( \lambda \ > 0 \) é o número médio de eventos ocorrendo no intervalo considerado.
\end{itemize}

Nestas condições, o comportamento de X pode ser descrito pela \textbf{Distribuição de Poisson} com parâmetro 
\( \lambda \). 

\begin{itemize}
	\item Notação: \(X \sim Po( \lambda ) \)
	\item Média de X: \( E(X) = \lambda \)
	\item Variância de X: \( V(X) = \lambda \)
\end{itemize}

\section{Modelos Probabilísticos para Variáveis Aleatórias Contínuas}

Uma variável é considerada contínua quando pode assumir qualquer valor dentro de um intervalo, ou seja, se houver um número incontável de resultados possíveis, representados por um intervalo sobre o eixo real.

\subsection{Distribuição Polinomial}

\begin{itemize}
	\item Notação: \(X \sim Exp( \alpha ) \)
	\item Média de X: \( E(X) = \frac{1}{\alpha} \)
	\item Variância de X: \( V(X) = \frac{1}{\alpha^2} \)
\end{itemize}

Função de densidade de probabilidade (f.d.p.):

\[ f(x) =
  \begin{cases}
    \alpha e^{- \alpha x}    & \text{se } x>0 \text{ e } \alpha>0\\
    0 		& \text{ ,para quaisuqer outros valores}
  \end{cases}
\]

Probabilidade de X para um intervalo de \textbf{a} até \textbf{b}\footnote{Nesse cenário o fato de \textbf{a} e \textbf{b} estarem ou não no intervalo citado é irrelevante para o cálculo da probabilidade.}:

\[\beq{ P( a \leq X \leq b ) = e^{- \alpha a} - e^{- \alpha b} }\]


\subsection{Distribuição Normal}

Propriedades:
\begin{itemize}
	\item A distribuição é simétrica em torno da média, assim, as medidas de tendência central (média, mediana e moda) apresentam o mesmo valor;
 	\item A distribuição normal fica delimitada pelo seu desvio padrão e sua média. Cada combinação de valores de média e desvio padrão gera uma distribuição Normal diferente;
	\item A área sob a curva corresponde à 1;
	\item O ponto máximo da curva da distribuição normal ocorre quando x = \(\mu\), ou seja, em torno da média registra-se uma probabilidade maior de ocorrência. 
	\item À medida que nos afastamos da média, as probabilidades de ocorrência vão diminuindo.
\end{itemize}

Notação: \(X \sim N( \mu , \sigma^2 ) \)

Função de densidade de probabilidade (f.d.p.), para \( - \infty < x < \infty \):


\[\beq{ f(x) = \frac{1}{\sigma\sqrt{2\pi}} e^{\frac{-(x-\mu)^2}{2\sigma^2}{}} }\]

\subsubsection{Distribuição Normal Padrão}

O cálculo de probabilidades, para variáveis adequadamente descritas pela distribuição normal, é realizado por meio da distribuição normal padrão, também chamada de padronizada ou reduzida. A variável aleatória Z tem distribuição normal padrão, pois sua média é igual a zero e sua variância é igual a 1, ou seja, \( Z \sim N (0, 1) \).

Função de densidade de probabilidade (f.d.p.), para a Distribuição Normal Padrão (\( \mu = 0 \) e \( \sigma^2 = 1 \)):


\[\beq{ f(z) = \frac{1}{\sqrt{2\pi}} e^{\frac{-(z)^2}{2}{}} }\]

Redução do valor de uma variável aleatória X para a Distribuição Normal Padrão:

\[\beq{ z = \frac{x - \mu}{\sigma} }\]
