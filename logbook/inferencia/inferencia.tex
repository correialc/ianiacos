\chapter{Inferência Estatística}

A estimação é o processo que consiste em utilizar dados amostrais para estimar os valores de parâmetros populacionais desconhecidos. 

\section{Estimativa Pontual}

A \textbf{estimativa é pontual} quando a estatística amostral origina uma única estimativa do parâmetro, ou seja, um único valor para o parâmetro populacional que se quer estimar:

\begin{table}[h]
	\centering	
	\caption{Estimadores dos principais parâmetros populacionais}
	\label{tab:estimadores}
	\begin{tabular}{c|c} 
		Parâmetro	& Estimador	\\
		\hline
		\(\mu\)	& \( \bar{x} \)	\\			
		\(\sigma\) & s		\\
		\( p \)	& \( \hat{p} \)		
	\end{tabular}
\end{table}

Sendo \(\hat{p} = \frac{x}{n}\), onde \textbf{x} representa o número de itens na amostra que apresentam uma característica de interesse e \textbf{n} representa o tamanho da amostra.


\section{Estimativa Intervalar ou Intervalo de Confiança}

A \textbf{estimativa é intervalar} quando, a partir da estatística amostral, construímos um intervalo de valores possíveis no qual se admite, sob certa probabilidade, esteja contido o parâmetro populacional.

Um \textbf{intervalo de confiança} está associado a um grau de confiança que é uma
medida da nossa certeza de que o intervalo contém o parâmetro populacional. O grau de confiança é a probabilidade \(1 - \alpha\) de o intervalo de confiança conter o verdadeiro valor do parâmetro populacional (o grau de confiança é também chamado de nível de confiança ou de coeficiente de confiança).

O \textbf{erro máximo da estimativa} ou \textbf{margem de erro}, denotada por \textbf{E} é a diferença máxima provável entre a média amostral observada e a verdadeira média populacional.

\[E = z_\frac{\alpha}{2}\frac{\sigma}{\sqrt{n}}\]

\section{Determinação do Tamanho da Amostra}

A determinação do tamanho de uma amostra é um problema de grande importância, porque amostras desnecessariamente grandes acarretam desperdício de tempo e de dinheiro.

\subsection{Cálculo de n para a Média em População Infinita}

Quando o desvio padrão da população é conhecido\footnote{Aproximar sempre para o maior inteiro mais próximo.}:

\[n = (\frac{z_\frac{\alpha}{2}\sigma}{E})^2\]

Quando não é possível determinar o desvio padrão da população:
\begin{alineas}
	\item Utilizar a regra prática para estimar o desvio padrão da seguinte maneira: \( \frac{amplitude}{4} \)
	\item Realizar um estudo piloto, iniciando o processo de amostragem. Com base na coleção de pelo menos 31 valores amostrais selecionados aleatoriamente, calcular o desvio padrão amostral \textbf{s} e utiliza-lo no lugar de \(\sigma\).
\end{alineas}

\subsection{Cálculo de n e E para a Média em População Finita}

Se a população é finita é necessário modificar a margem de erro (E), com a inclusão de um fator de correção para população finita, sendo N o tamanho conhecido da população.

\[E = z\frac{\sigma}{\sqrt{n}}\sqrt{\frac{N-n}{N-1}}\]

\[n = \frac{N\sigma^2z^2}{(N-1)E^2+\sigma^2z^2} \]

\subsection{Intervalo de Confiança da Média para \(\sigma\) Desconhecido}

Vimos que a distribuição Z é adequada quando o desvio padrão populacional (\(\sigma\)) é conhecido. Porém, se não conhecemos (\(\sigma\)), podemos utilizar a distribuição t-Student. O parâmetro da distribuição t-Student é o grau de liberdade (g.l.), definido
por n-1.

\[E = t_\frac{\alpha}{2}\frac{s}{\sqrt{n}}\]

Onde \(t_\frac{\alpha}{2}\) tem \(n-1\) graus de liberdade.

\section{Intervalo de Confiança para a Proporção}

Assim como a média amostral é a melhor estimativa pontual para a média populacional, a proporção amostral (\(\hat{p}\)) é a melhor estimativa pontual da proporção populacional (\({p}\)), supondo que \(n.\hat{p}\geq5\) e \(n.(1-\hat{p})\geq5\)\footnote{Em virtude da construção de intervalos de confiança para a proporção utilizar a distribuição normal como aproximação da distribuição de proporções amostrais}.

\[E = z_\frac{\alpha}{2}\sqrt{\frac{\hat{p}(1-\hat{p})}{n}}\]