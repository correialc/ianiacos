\chapter{Introdução}

A estatística envolve técnicas para coletar, organizar,
descrever, analisar e interpretar dados provenientes de
experimentos ou vindos de outros estudos
observacionais.

\section{Conceitos básicos}

\begin{itemize}
	\item População: conjunto de todos os resultados, respostas, medidas ou contagens a serem estudados;
	\item Censo: conjunto de dados relativos a todos os elementos de uma população;
	\item Dados: informações provenientes de observações, contagens, medidas ou respostas;
	\item Amostra: subconjunto de elementos de uma população;
	\item Amostragem: processo de seleção da amostra.
	\item Parâmetro: medida que descreve numericamente uma característica da população.
	\item Estatística: medida que descreve numericamente uma característica da amostra.
\end{itemize}

\section{Tipos de Dados}

\begin{itemize}
	\item Qualitativos (ou categóricos): representam atributos, qualidades, características. Podem ser subdivididos em \textbf{nominais}(quando não possuem um ordem ou hierarquia) e \textbf{ordinais} (quando possuem uma ordem ou hierarquia);
	\item Quantitativos (ou não categóricos): consistem em medidas ou contagens. Podem ser subdivididos em \textbf{discretos} (conjunto finito ou enumerável de valores possíveis) ou \textbf{contínuos} (conjunto infinito de valores possíveis).
\end{itemize}