\chapter{Estatística Descritiva}

Compreende a organização, o resumo e, em geral, a simplificação de informações que podem ser muito complexas.

\section{Medidas de Tendência Central}

Mostram o valor em torno do qual os dados tendem a agrupar-se.

\subsection{Média Aritmética}

Representa o valor médio da amostra ou da população. Por estar diretamente relacionada aos valores dos dados analisados, a média é sensível a grande diferenças de valor. 

Notações:

\(\bar{x}\) Notação da média aritmática \emph{da amostra}

\(\mu\) Notação da média aritmática \emph{da população}

\[\beq{ \bar{x} = \frac{\sum_{i=1}^{n} x_i}{n} }\] 

\subsection{Mediana}

Representa o valor que divide a amostra ou a população em duas partes de igual tamanho. Diferente do que ocorre com a média aritmética, a mediana é uma medida posicional com relação aos dados, ocupando uma posição central mesmo que os dados analisados possuam grandes diferenças de valor.

Notação: \(\tilde{x}\)

Seja \(n\) o tamanho da amostra:
\begin{itemize}
	\item O conjunto de dados da amostra precisa estar ordenado.
	\item Se \(n\) for ímpar, a mediana será o valor que ocupa a posição \(\frac{n + 1}{2}\)
	\item Se \(n\) for par, a mediana será a média dos valores que ocupam as posições \(\frac{n}{2}\) e \(\frac{n + 2}{2}\)
\end{itemize}

\section{Medidas de Posição Relativa}

Mostram pontos de corte na distribuição relativa dos dados da amostra:

\begin{itemize}
	\item Quartis: três pontos (Q1, Q2 e Q3) que dividem a amostra em 4 partes de igual tamanho, a 25\% (Q1), 50\% (Q2 ou mediana) e 75\% (Q3);
	\item Decis: 9 pontos (D1 a D9) que dividem a amostra em 10 partes  de igual tamanho;
	\item Percentis: 99 pontos (P1 a P99) que dividem a amostra em 100 partes  de igual tamanho. Existem correspondências entre percentis, quartis e decis.
\end{itemize}

\subsection{Percentil}

\[\beq{ L = \frac{k}{100}n }\]

Onde:

\(n\) = tamanho da amostra

\(k\) = percentil que se deseja calcular	

\(L\) = posição do percentil na amostra

\begin{itemize}
    \item Se \(L\) é um número inteiro, o percentil \(P_k\) pode ser obtido pela média 			entre os valores que ocupam as posições \(L\) e \(L+1\).
    \item Se \(L\) é um número decimal, devemos arredondar \(L\) para o valor interiro 			mais próximo \emph{a maior} e o percentil \(P_k\) será obtido pelo valor que ocupa a 		posição \(L\) já arredondada.      
\end{itemize}

\subsection{Escore z}

O Escore Z é uma medida que expressa, em unidades de desvio padrão, o quanto um determinado número está distante da média.

\[\beq{ z = \frac{x-\bar{x}}{s} }\]

\(z\) = \emph{escore z}

\(x\) = valor na amostra a partir do qual se quer medir o \emph{escore z}

\(\bar{x}\) = média aritmética da amostra

\(s\) = desvio padrão da amostra

\section{Medidas de Dispersão ou Variabilidade}

Mostram o grau de afastamento dos valores observados em relação ao ponto central da distribuição dos dados.

\subsection{Amplitude}

É definida como sendo a diferença entre o maior e o menor valor
do conjunto de dados. É representada pela letra \textbf{A}.

\subsection{Variância e Desvio Padrão}

Indicam o grau de afastamento dos valores observados em relação à média aritmética. O desvio padrão possui a mesma unidade de medida dos dados observados, enquanto a variância não, mas ambos representam a mesma informação. A tabela \ref{tab:notacoes-var-dp} apresenta as notações da variância e do desvio padrão.

\begin{table}[h]
	\centering	
	\caption{Notações de Variância e Desvio Padrão}
	\label{tab:notacoes-var-dp}
	\begin{tabular}{l|cc} 
		Medidas 		& População 	& Amostra 		\\
		\hline
		Variância		& \(\sigma^2 \)	& \(s^2\)	\\
		Desvio Padrão	& \(\sigma \)	& \(s\)
	\end{tabular}
\end{table}

Variância (população):

\[\beq{ \sigma^2=\frac{\sum_{i=1}^{n} (x_i - \bar{x})^2}{n} }\]

Variância (amostra):

\[\beq{ s^2=\frac{\sum_{i=1}^{n} (x_i - \bar{x})^2}{n-1} }\]

Desvio Padrão (população):

\[\beq{ \sigma=\sqrt{\frac{\sum_{i=1}^{n} (x_i - \bar{x})^2}{n}} }\]

Desvio Padrão (amostra):

\[\beq{ s=\sqrt{\frac{\sum_{i=1}^{n} (x_i - \bar{x})^2}{n-1}} }\]

\subsection{Coeficiente de Variação}

O desvio padrão embora seja a medida de dispersão mais utilizada, mede a dispersão em termos absolutos. O coeficiente de variação (CV) mede a dispersão em termos relativos. O coeficiente de variação é adimensional (não tem unidade de medida), tornando-se útil quando queremos comparar a variabilidade de observações com diferentes unidades de medidas. O CV é expresso em porcentagem. É zero quando não houver variabilidade entre os dados. Quanto menor o coeficiente de variação, mais
homogêneo é o conjunto de dados.

\[\beq{ CV=\frac{s}{\bar{x}} 100 }\]

A tabela \ref{tab:coeficiente-variabilidade} evidencia uma possível classificação de homogeneidade de acordo com o coeficiente de variabilidade.

\begin{table}[h]
	\centering	
	\caption{Classificação da Homogeneidade a partir do CV}
	\label{tab:coeficiente-variabilidade}
	\begin{tabular}{l|c} 
		CV 		& Homogeneidade	\\
		\hline
		< 10\%		& muito alta	\\
		de 10\% a 20\%	& alta		\\
		de 20\% a 30\%	& média		\\
		> 30\%			& baixa		
	\end{tabular}
\end{table}

