\chapter{Estatística Descritiva}

\section{Medidas de Tendência Central}

Mostram o valor em torno do qual os dados tendem a agrupar-se.

\subsection{Média Aritmética}

Notações:

\(\bar{x}\) Notação da média aritmática \emph{da amostra}

\(\mu\) Notação da média aritmática \emph{da população}

\[\bar{x} = \sum_{i=1}^{n} x_i \] 

\subsection{Mediana}

Notação: \(\tilde{x}\)

Seja \(n\) o tamanho da amostra:
\begin{itemize}
	\item O conjunto de dados da amostra precisa estar ordenado.
	\item Se \(n\) for ímpar, a mediana será o valor que ocupa a posição \(\frac{n + 1}{2}\)
	\item Se \(n\) for par, a mediana será a média dos valores que ocupam as posições \(\frac{n}{2}\) e \(\frac{n + 2}{2}\)
\end{itemize}

\section{Medidas de Posição Relativa}

Mostram pontos de corte na distribuição relativa dos dados da amostra.

\subsection{Percentil}

\[L = \frac{k}{100} \times n\]

Onde:

\(n\) = tamanho da amostra

\(k\) = percentil que se deseja calcular	

\(L\) = posição do percentil na amostra

\begin{itemize}
    \item Se \(L\) é um número inteiro, o percentil \(P_k\) pode ser obtido pela média 			entre os valores que ocupam as posições \(L\) e \(L+1\).
    \item Se \(L\) é um número decimal, devemos arredondar \(L\) para o valor interiro 			mais próximo \emph{a maior} e o percentil \(P_k\) será obtido pelo valor que ocupa a 		posição \(L\) já arredondada.      
\end{itemize}

\subsection{Escore z}

\[z = \frac{x-\bar{x}}{s}\]

\(z\) = \emph{escore z}

\(x\) = valor na amostra a partir do qual se quer medir o \emph{escore z}

\(\bar{x}\) = média aritmética da amostra

\(s\) = desvio padrão da amostra

\section{Medidas de Dispersão ou Variabilidade}

Mostram o grau de afastamento dos valores observados em relação ao ponto central da distribuição dos dados.

\subsection{Variância e Desvio Padrão}

\begin{table}[h]
	\centering	
	\caption{Notações de Variância e Desvio Padrão}
	\begin{tabular}{l|cc} 
		Medidas 		& População 	& Amostra 		\\
		\hline
		Variância		& \(\sigma^2 \)	& \(s^2\)	\\
		Desvio Padrão	& \(\sigma \)	& \(s\)
	\end{tabular}
\end{table}

Variância (população):

\[ \sigma^2=\frac{\sum_{i=1}^{n} (x_i - \bar{x})^2}{n} \]

Variância (amostra):

\[ s^2=\frac{\sum_{i=1}^{n} (x_i - \bar{x})^2}{n-1} \]

Desvio Padrão (população):

\[ \sigma=\sqrt{\frac{\sum_{i=1}^{n} (x_i - \bar{x})^2}{n}} \]

Desvio Padrão (amostra):

\[ s=\sqrt{\frac{\sum_{i=1}^{n} (x_i - \bar{x})^2}{n-1}} \]

\subsection{Coeficiente de Variação}

\[ CV=\frac{s}{\bar{x}} \times 100 \]